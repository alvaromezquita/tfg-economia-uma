
%%% PLANTILLA PARA TFG, FACULTAD DE C. ECONOMICAS Y EMPRESARIALES UMA %%%

%%%%%%%%%%%%%%%%%%%%%%%%%%%%%%%%%%%%%%%%%%%%%%%%%%%%%%%%%%%%%%%%%%%%%%%%%
%% This file is part of PLANTILLA_TFG_UMA.
%% 
%% PLANTILLA_TFG_UMA is distributed under the LaTeX Project Public License, version 1.3c or later.
%% See https://www.latex-project.org/lppl/ for the full text of the license.
%%
%% This work may be distributed and/or modified under the conditions of the LaTeX Project Public License.
%% 
%% The Current Maintainer of this work is Álvaro Mezquita Martínez.
%%%%%%%%%%%%%%%%%%%%%%%%%%%%%%%%%%%%%%%%%%%%%%%%%%%%%%%%%%%%%%%%%%%%%%%%%

%%%%%%%%%%%%%%%%%%%%%%%%%%%%%%%%%%%%%%%%%%%%%%%%%%%%%
%%% LEER LEEME.TXT ANTES DE UTILIZAR LA PLANTILLA %%%
%%%%%%%%%%%%%%%%%%%%%%%%%%%%%%%%%%%%%%%%%%%%%%%%%%%%%

%%%%%%%%%%%%%%%%%
%%% DOCUMENTO %%% 
%%%%%%%%%%%%%%%%%


\documentclass[13pt,a4paper]{report}

% Importando configuración de config.tex

%%%%%%%%%%%%%%%%%%%%%%%%%%%%%%
%% ARCHIVO DE CONFIGURACIÓN %%
%%%%%%%%%%%%%%%%%%%%%%%%%%%%%%

% Este archivo es mejor no tocarlo si no se sabe lo que se hace. Sin embargo, cualquier cambio en la normativa de estilo, o alguna configuración extra necesaria para un trabajo específico necesita tocar cosas de aquí. 
% De nuevo, si no eres experimentado, mejor dejarlo así. 

%% PAQUETES %% 
% Por lo general de aquí no quieres tocar nada, como mucho añadir algun paquete como los recomendados de LEEME.txt

\usepackage[utf8]{inputenc}
\usepackage{graphicx}
\usepackage{fancyhdr} % Configurar encabezado y pie de página
\usepackage{setspace}
\usepackage[spanish]{babel} % Establece el idioma en español
\usepackage{hyperref} 
\usepackage{apacite}
\usepackage{fontspec}
\usepackage{titlesec} 
\usepackage{amssymb}
\usepackage{amsfonts}
\usepackage{amsmath}
\usepackage{boxedminipage}
\usepackage{graphics} 
\usepackage{graphpap}    
\usepackage{booktabs}
\usepackage{natbib} 
\usepackage{fvextra}
\usepackage{float}
\usepackage{caption} 
\usepackage{anyfontsize}
\usepackage{chngcntr}
\usepackage{titletoc}
\usepackage{tocloft}
\usepackage{capt-of}

%% AJUSTES DE LOS PAQUETES %% 
% En este apartado se configura el documento para que replique el formato exigido por la Facultad de Ciencias Economicas y Empresariales de la Universidad de Málaga. De nuevo, no se recomienda la modificación de estos aparatados, pero puede ser necesario en un determinado contexto.

% Para evitar que el parrafo empiece con sangría
\setlength{\parindent}{0pt}

% Ajustar espacio entre las entradas del índice
\setlength{\cftparskip}{5pt}

% Cargar la fuente personalizada
\setmainfont[
  Path = ./,
  Extension = .ttf,
  UprightFont = GARA, % Fuente normal
  ItalicFont = GARA-IT, % Fuente en cursiva
  BoldFont = GARA-BD % Fuente en negrita
]{GARA}

% Poner en cursiva la descripción
\captionsetup{
  font={it} % Definimos explícitamente la fuente aquí
}

% Formateando los capítulos según normas de estilo 
\titleformat{\section}
  {\fontsize{16}{18}\selectfont\bfseries} % Negrita, tamaño 16pt, todo en mayúsculas
  {}                         % Este campo vacío oculta el número de la sección
  {0pt}                      % Sin espacio entre el margen y el título
  {}

% Formateando los epígrafes según normas de estilo 
\titleformat{\subsection}
  {\fontsize{14}{16}\selectfont\bfseries} % Negrita, tamaño 14pt
  {\thesubsection}{1em}{} 
\titlespacing*{\section}
  {0pt}{12pt}{6pt}

% Formateando los subepígrafes según normas de estilo 
\titleformat{\subsubsection}
  {\fontsize{14}{16}\selectfont\itshape} % Cursiva, tamaño 14pt
  {\thesubsubsection}{1em}{}  
\titlespacing*{\subsection}
  {0pt}{12pt}{6pt}

\renewcommand\normalsize{\fontsize{13}{15.6}\selectfont}
\normalsize

% Márgenes
\usepackage[a4paper, left=3cm, right=3cm, top=2.5cm, bottom=2.5cm]{geometry}

% Interlineado
\setstretch{1.2}

% Encabezado y pie de página
\pagestyle{fancy}
\fancyhf{}
\fancyhead[R]{\includegraphics[width=4cm]{logo.png}} % Logo de la Facultad a la derecha
\fancyfoot[C]{\thepage} % Número de página centrado en el pie de página

% Encabezado y pie de pagina para paginas especiales (índice, bibliografia...)
\fancypagestyle{plain}{
  \fancyhf{} % Limpiar encabezados y pies de página
  \fancyhead[R]{\includegraphics[width=4cm]{logo.png}} % Imagen en el encabezado izquierdo
  \fancyfoot[C]{\thepage} % Número de página en el pie de página
}

\renewcommand{\headrulewidth}{0pt}
\setcounter{tocdepth}{3}
\counterwithout{section}{chapter}
\counterwithin{table}{section}
\numberwithin{figure}{section} % Numeración de figuras por sección
\numberwithin{table}{section} % Numeración de tablas por sección 

% Estilo para las secciones (en negrita)
\titlecontents{section}
  [0em] % Espacio izquierdo
  {\bfseries} % Formato de la etiqueta
  {\thecontentslabel\hspace{1em}} % Formato del número de sección
  {} % Espacio entre la etiqueta y el título
  {\titlerule*[0.5pc]{.}\contentspage} % Formato del título

% Estilo para las subsecciones
\titlecontents{subsection}
  [1.5em] % Espacio izquierdo
  {\normalfont} % Formato de la etiqueta
  {\thecontentslabel\hspace{1em}} % Formato del número de subsección
  {} % Espacio entre la etiqueta y el título
  {\titlerule*[0.5pc]{.}\contentspage} % Formato del título

\titlecontents{subsubsection}
  [3em] % Indentación para subsubsección
  {\thecontentslabel\hspace{1em}} % Formato del número de subsubsección
  {} % Espacio entre la etiqueta y el título
  {} % No se usa prefijo antes del título
  {\titlerule*[0.5pc]{.}\contentspage} % Puntos de conexión y número de página

  
% Definir el formato para las secciones en el TOC (table of contents)
\titlecontents{section}
  [0em] % Espacio izquierdo
  {} % Formato de la etiqueta (vacío para eliminar el número)
  {} % Formato del número de sección (vacío)
  {} % Espacio entre la etiqueta y el título
  {\titlerule*[0.5pc]{.}\contentspage} % Formato del título
  
\renewcommand{\cftsecdotsep}{4} % Ajusta para secciones
\renewcommand{\cftsubsecdotsep}{4} % Ajusta para subsecciones
\renewcommand{\cftsubsubsecdotsep}{4} % Ajusta para subsubsecciones
\newcommand{\fuente}[1]{\fontsize{14}{16}\selectfont\itshape}

\begin{document}

% Portada
\begin{titlepage}
    \centering 
    \begin{figure}[h!]
        \centering
        \includegraphics[width=0.5\textwidth]{logo.png}
    \end{figure}
    
    \vspace*{1cm} 
    \Huge
    \textbf{Trabajo Fin de Grado}
    \vspace*{1cm}
    
    \Huge
    
    %Titulo del trabajo
    \textbf{Plantilla para el TFG con \LaTeX}
    
    \vspace{0.5cm}
    \LARGE 
    
    % Nombre del alumno
    Álvaro Mezquita Martínez
    
    \vspace{0.5cm} 

    Curso 2024/2025
        
    \vspace{0.5cm}

    % Grado del alumno
    Grado en Economía
    
    \vfill

    \raggedleft     
    Tutor:  \makebox[5cm]{\dotfill} %Aquí debes escribir el nombre completo de tu tutor  
    \\
    Departamento:  \makebox[5cm]{\dotfill} %Aquí debes escribir el nombre del departamento pertinente
    
    \vspace{5cm}
\end{titlepage}

% Declaración de originalidad

\newpage

\fancyhf{} % Limpia encabezado y pie de página
\fancyhead[R]{\includegraphics[width=4cm]{logo.png}} % Solo imagen en el encabezado derecho
\thispagestyle{fancy} % Aplica este estilo específico a la página

\noindent {\fontsize{16}{19.2}\selectfont \textbf{Declaración de originalidad y de visto bueno del tutor o tutora}} % "Declaración de originalidad" formateado

\vspace{12pt}

Yo, \makebox[2cm]{\dotfill}, con DNI  \makebox[2cm]{\dotfill}, declaro:  %Eliminar \makebox[]{\dotfill}

\vspace{12pt}

Que este Trabajo Fin de Grado que presento para su evaluación y defensa es original, y que todas las fuentes utilizadas para su realización han sido debidamente citadas en el mismo. 

\vspace{12pt}

Y que cuento con el visto bueno de mi tutor/a para solicitar la defensa en este llamamiento.
\begin{flushright}
    Málaga a  \makebox[2cm]{\dotfill} \\ %Eliminar \makebox[]{\dotfill}
    \vspace{5cm}
     Firmado: \makebox[5cm]{\dotfill} %Eliminar \makebox[]{\dotfill}
\end{flushright}



\newpage
\pagenumbering{roman} % Numeración romana para el índice y resumen

% Índice (NO TOCAR)
\renewcommand{\contentsname}{ÍNDICE} 
\clearpage
\tableofcontents

% Resumen, Palabras Clave y Título en Inglés 
\newpage 

% Encabezado y pie de página
\pagestyle{fancy}
\fancyhf{}
\fancyhead[R]{\includegraphics[width=4cm]{logo.png}} % Logo de la Facultad a la derecha
\fancyfoot[C]{\fontsize{13pt}{15pt}\selectfont\thepage} % Número de página centrado en el pie de página

\pagenumbering{arabic} %Cambia la numeracion a romana

\section{Título en inglés}

\vspace{12pt}

Aquí escribes la traducción de tu título al inglés (no lo escribas todo en mayúsculas, no se aceptan versiones libres, los dos títulos deben coincidir).

\vspace{12pt}

\section{Resumen}

\vspace{12pt}

Aquí escribes tu resumen de no más de 200 palabras.

\vspace{12pt}

\section{Palabras clave}

\vspace{12pt}

% Palabras clave de tu trabajo
Aquí escribes las palabras clave que definen a tu TFG, no más de cinco.

\newpage  

% A la hora de utilizar \section y \subsection, ten en cuenta que \section debe ir solo y \subsection debe ir junto con el comando \addcontentsline para incluirlo correctamente en el índice

\section{Utilice esto para el título del capítulo} 

\subsection*{Utilice esto para el título del epígrafe} 
\addcontentsline{toc}{subsection}{Utilice esto para el título del epígrafe} % Añadir al índice


\subsubsection*{Utilice esto para el título del subepígrafe} 
\addcontentsline{toc}{subsubsection}{Utilice esto para el título del subepígrafe} % Añadir al índice

\begin{center}
    \centering % Centra el caption
    \captionof{figure}{Logo de la facultad} % Nombre de la figura
    \includegraphics[width=\linewidth]{logo.png} % Imagen
    \label{fig:imagen_facultad} % Etiqueta para referenciar
    \par % Termina la alineación centrada
    \raggedright % Restablece la alineación a la izquierda 
    {\fontsize{12}{14}\selectfont\textit{Fuente: Facultad de CC. Económicas y EE. (2017).}} % Fuente en 12 puntos y en itálico
\end{center}

\vspace{12pt}

Este es un ejemplo de referencia de la figura/imagen \ref{fig:imagen_facultad}, y este otro es para referenciar la bibliografía \citep{tfg_sample}. Aquí otro ejemplo con un artículo \citep{baccini2007edgeworth}

\vspace{12pt}

Para enseñar el potencial de \LaTeX, escribamos una función de producción Cobb-Douglass \citep{cobb_theory_1928}: 

\begin{equation}
    Y = f(L,K) = A K^\alpha L^{1-\alpha} \label{eq: cobb-d}
\end{equation}

\vspace{12pt}

Como se aprecia en la ecuación \ref{eq: cobb-d}, \LaTeX\ destaca por su capacidad para renderizar ecuaciones con gran precisión y calidad tipográfica. Además, su sistema de referencias cruzadas permite citar ecuaciones de manera sencilla y consistente a lo largo del documento, lo que contribuye a una presentación más profesional y organizada, ideal para un TFG.
\section{Conclusiones} 

\vspace{12pt}

Todo TFG debe finalizar con unas conclusiones, que recojan tu análisis y aportaciones.


% Bibliografía (NO TOCAR)

\bibliographystyle{apalike} % Establecer formato APA 7a Edición
\renewcommand{\bibname}{\fontsize{16}{20}\selectfont Bibliografía} % Nombre y formato de la bibliografía 
\renewcommand{\bibfont}{\fontsize{13}{15}\selectfont} % Cambiar a 13 puntos el tamaño del texto
\bibliography{bib_tfg} % Llamada a la bibliografía
\addcontentsline{toc}{section}{Bibliografía} % Añadir al índice 

\vspace{12pt}

\section{Anexo Legislación}

\vspace{12pt}

\href{https://biblioguias.uma.es/citasybibliografia/Legislacion}{https://biblioguias.uma.es/citasybibliografia/Legislacion} 

\vspace{12pt}

En caso de que el TFG contenga un cuantioso número de normativa citada en su texto, se recomienda elaborar un Anexo de Legislación y un Anexo de jurisprudencia, situados tras la Bibliografía o Lista de referencias, y ordenados cronológicamente de forma ascendente (normas/sentencias más antiguas primero). La legislación y la jurisprudencia citada en el cuerpo del texto no figura en ningún caso en la Bibliografía o Lista de referencias sino en los mencionados anexos.
(para la creación de este anexo de legislación siga las indicaciones del enlace:

\vspace{12pt}

\href{https://uc3m.libguides.com/guias\_tematicas/citas\_bibliograficas/legislacion}{https://uc3m.libguides.com/guias\_tematicas/citas\_bibliograficas/legislacion}
 ) 

 \vspace{12pt}

y el formato de párrafo de la bibliografía.

\vspace{12pt}

\section{Anexos}

(En caso necesario irán detrás de la bibliografía y no computan para el mínimo de páginas exigidas del TFG)

\end{document}
