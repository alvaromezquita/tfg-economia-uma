
%%%%%%%%%%%%%%%%%%%%%%%%%%%%%%
%% ARCHIVO DE CONFIGURACIÓN %%
%%%%%%%%%%%%%%%%%%%%%%%%%%%%%%

% Este archivo es mejor no tocarlo si no se sabe lo que se hace. Sin embargo, cualquier cambio en la normativa de estilo, o alguna configuración extra necesaria para un trabajo específico necesita tocar cosas de aquí. 
% De nuevo, si no eres experimentado, mejor dejarlo así. 

%% PAQUETES %% 
% Por lo general de aquí no quieres tocar nada, como mucho añadir algun paquete como los recomendados de LEEME.txt

\usepackage[utf8]{inputenc}
\usepackage{graphicx}
\usepackage{fancyhdr} % Configurar encabezado y pie de página
\usepackage{setspace}
\usepackage[spanish]{babel} % Establece el idioma en español
\usepackage{hyperref} 
\usepackage{apacite}
\usepackage{fontspec}
\usepackage{titlesec} 
\usepackage{amssymb}
\usepackage{amsfonts}
\usepackage{amsmath}
\usepackage{boxedminipage}
\usepackage{graphics} 
\usepackage{graphpap}    
\usepackage{booktabs}
\usepackage{natbib} 
\usepackage{fvextra}
\usepackage{float}
\usepackage{caption} 
\usepackage{anyfontsize}
\usepackage{chngcntr}
\usepackage{titletoc}
\usepackage{tocloft}
\usepackage{capt-of}

%% AJUSTES DE LOS PAQUETES %% 
% En este apartado se configura el documento para que replique el formato exigido por la Facultad de Ciencias Economicas y Empresariales de la Universidad de Málaga. De nuevo, no se recomienda la modificación de estos aparatados, pero puede ser necesario en un determinado contexto.

% Para evitar que el parrafo empiece con sangría
\setlength{\parindent}{0pt}

% Ajustar espacio entre las entradas del índice
\setlength{\cftparskip}{5pt}

% Cargar la fuente personalizada
\setmainfont[
  Path = ./,
  Extension = .ttf,
  UprightFont = GARA, % Fuente normal
  ItalicFont = GARA-IT, % Fuente en cursiva
  BoldFont = GARA-BD % Fuente en negrita
]{GARA}

% Poner en cursiva la descripción
\captionsetup{
  font={it} % Definimos explícitamente la fuente aquí
}

% Formateando los capítulos según normas de estilo 
\titleformat{\section}
  {\fontsize{16}{18}\selectfont\bfseries} % Negrita, tamaño 16pt, todo en mayúsculas
  {}                         % Este campo vacío oculta el número de la sección
  {0pt}                      % Sin espacio entre el margen y el título
  {}

% Formateando los epígrafes según normas de estilo 
\titleformat{\subsection}
  {\fontsize{14}{16}\selectfont\bfseries} % Negrita, tamaño 14pt
  {\thesubsection}{1em}{} 
\titlespacing*{\section}
  {0pt}{12pt}{6pt}

% Formateando los subepígrafes según normas de estilo 
\titleformat{\subsubsection}
  {\fontsize{14}{16}\selectfont\itshape} % Cursiva, tamaño 14pt
  {\thesubsubsection}{1em}{}  
\titlespacing*{\subsection}
  {0pt}{12pt}{6pt}

\renewcommand\normalsize{\fontsize{13}{15.6}\selectfont}
\normalsize

% Márgenes
\usepackage[a4paper, left=3cm, right=3cm, top=2.5cm, bottom=2.5cm]{geometry}

% Interlineado
\setstretch{1.2}

% Encabezado y pie de página
\pagestyle{fancy}
\fancyhf{}
\fancyhead[R]{\includegraphics[width=4cm]{logo.png}} % Logo de la Facultad a la derecha
\fancyfoot[C]{\thepage} % Número de página centrado en el pie de página

% Encabezado y pie de pagina para paginas especiales (índice, bibliografia...)
\fancypagestyle{plain}{
  \fancyhf{} % Limpiar encabezados y pies de página
  \fancyhead[R]{\includegraphics[width=4cm]{logo.png}} % Imagen en el encabezado izquierdo
  \fancyfoot[C]{\thepage} % Número de página en el pie de página
}

\renewcommand{\headrulewidth}{0pt}
\setcounter{tocdepth}{3}
\counterwithout{section}{chapter}
\counterwithin{table}{section}
\numberwithin{figure}{section} % Numeración de figuras por sección
\numberwithin{table}{section} % Numeración de tablas por sección 

% Estilo para las secciones (en negrita)
\titlecontents{section}
  [0em] % Espacio izquierdo
  {\bfseries} % Formato de la etiqueta
  {\thecontentslabel\hspace{1em}} % Formato del número de sección
  {} % Espacio entre la etiqueta y el título
  {\titlerule*[0.5pc]{.}\contentspage} % Formato del título

% Estilo para las subsecciones
\titlecontents{subsection}
  [1.5em] % Espacio izquierdo
  {\normalfont} % Formato de la etiqueta
  {\thecontentslabel\hspace{1em}} % Formato del número de subsección
  {} % Espacio entre la etiqueta y el título
  {\titlerule*[0.5pc]{.}\contentspage} % Formato del título

\titlecontents{subsubsection}
  [3em] % Indentación para subsubsección
  {\thecontentslabel\hspace{1em}} % Formato del número de subsubsección
  {} % Espacio entre la etiqueta y el título
  {} % No se usa prefijo antes del título
  {\titlerule*[0.5pc]{.}\contentspage} % Puntos de conexión y número de página

  
% Definir el formato para las secciones en el TOC (table of contents)
\titlecontents{section}
  [0em] % Espacio izquierdo
  {} % Formato de la etiqueta (vacío para eliminar el número)
  {} % Formato del número de sección (vacío)
  {} % Espacio entre la etiqueta y el título
  {\titlerule*[0.5pc]{.}\contentspage} % Formato del título
  
\renewcommand{\cftsecdotsep}{4} % Ajusta para secciones
\renewcommand{\cftsubsecdotsep}{4} % Ajusta para subsecciones
\renewcommand{\cftsubsubsecdotsep}{4} % Ajusta para subsubsecciones
\newcommand{\fuente}[1]{\fontsize{14}{16}\selectfont\itshape}