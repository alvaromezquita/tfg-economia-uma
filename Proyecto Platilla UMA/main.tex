
%% This file is part of PLANTILLA_TFG_UMA.
%% 
%% PLANTILLA_TFG_UMA is distributed under the LaTeX Project Public License, version 1.3c or later.
%% See https://www.latex-project.org/lppl/ for the full text of the license.
%%
%% This work may be distributed and/or modified under the conditions of the LaTeX Project Public License.
%% 
%% The Current Maintainer of this work is Álvaro Mezquita Martínez.



\documentclass[13pt,a4paper]{report}

%Paquetes. 

% "main.tex", está equipado con muchos paquetes no solo para generar ese formateo tan radical, sino también para permitir al usuario adaptarse a diferentes situaciones en las que necesite "esa libreria" o "ese comando".

%Algunos son innecesarios para la mayoría de trabajos, pero es altamente recomendable NO MODIFICARLOS. 

\usepackage[utf8]{inputenc}
\usepackage{graphicx}
\usepackage{fancyhdr}
\usepackage{setspace}
\usepackage[spanish]{babel}
%\usepackage{csquotes}
\usepackage{hyperref}
\usepackage{apacite}
\usepackage{setspace}
\usepackage{fontspec}
\usepackage{titlesec} 
\usepackage{amssymb}
\usepackage{amsfonts}
\usepackage{amsmath}
\usepackage{boxedminipage}
\usepackage{fancyhdr}
\usepackage{graphicx}
\usepackage{graphics} 
\usepackage{graphpap}    
%\usepackage[utf8x]{inputenc}   
\usepackage{booktabs}
%\usepackage[autostyle]{csquotes} 
%\usepackage{floatrow}
\usepackage{natbib} 
\usepackage{fvextra}
\usepackage{float}
\usepackage{caption} 
%\usepackage[font=12]{caption} % Para cambiar el tamaño del caption
\usepackage{anyfontsize}
\usepackage{chngcntr}
\usepackage{titletoc}
\usepackage{tocloft}
\usepackage{capt-of}
\usepackage{tikz}


%Ajustes del paquete Verbatim para mostrar códigos
\DefineVerbatimEnvironment{verbatim}{Verbatim}{breaklines=true}
\setcounter{MaxMatrixCols}{10}
\graphicspath{{Intro/}{Ch1/}{Ch2/}{Ch3/}{Ch4/}{Ch5/}{Ch6/}{Ch7/}{Ch8/}{Ch9/}}

% Teoremas
\newtheorem{theorem}{Theorem}
\newtheorem{acknowledgement}[theorem]{Acknowledgement}
\newtheorem{algorithm}[theorem]{Algorithm}
\newtheorem{axiom}[theorem]{Axiom}
\newtheorem{case}[theorem]{Case}
\newtheorem{claim}[theorem]{Claim}
\newtheorem{conclusion}[theorem]{Conclusion}
\newtheorem{condition}[theorem]{Condition}
\newtheorem{conjecture}[theorem]{Conjecture}
\newtheorem{corollary}[theorem]{Corollary}
\newtheorem{criterion}[theorem]{Criterion}
\newtheorem{definition}[theorem]{Definition}
\newtheorem{example}[theorem]{Example}
\newtheorem{exercise}[theorem]{Exercise}
\newtheorem{lemma}[theorem]{Lemma}
\newtheorem{notation}[theorem]{Notation}
\newtheorem{problem}[theorem]{Problem}
\newtheorem{proposition}[theorem]{Proposition}
\newtheorem{remark}[theorem]{Remark}
\newtheorem{solution}[theorem]{Solution}
\newtheorem{summary}[theorem]{Summary}

% Ajustes del archivo TeX. 
% En este apartado se configura el documento para que replique el formato exigido por la Facultad de Ciencias Economicas y Empresariales de la Universidad de Málaga. De nuevo, no se recomienda la modificación de estos aparatados

% Para evitar que el parrafo empiece con sangría
\setlength{\parindent}{0pt}

% Ajustar espacio entre las entradas del índice
\setlength{\cftparskip}{5pt}

% Cargar la fuente personalizada
\setmainfont[
  Path = ./,
  Extension = .ttf,
  UprightFont = GARA,
  ItalicFont = GARA-IT,
  BoldFont = GARA-BD
]{GARA}


\captionsetup{
  font={it} % Definimos explícitamente la fuente aquí
}

\titleformat{\section}
  {\fontsize{16}{18}\selectfont\bfseries\MakeUppercase} % Negrita, tamaño 16pt, todo en mayúsculas
  {}                         % Este campo vacío oculta el número de la sección
  {0pt}                      % Sin espacio entre el margen y el título
  {}

% Redefinir la subsección
\titleformat{\subsection}
  {\fontsize{14}{16}\selectfont\bfseries} % Negrita, tamaño 14pt
  {\thesubsection}{1em}{} 
\titlespacing*{\section}
  {0pt}{12pt}{6pt}

% Redefinir la subsubsección
\titleformat{\subsubsection}
  {\fontsize{14}{16}\selectfont\itshape} % Cursiva, tamaño 14pt
  {\thesubsubsection}{1em}{}  
\titlespacing*{\subsection}
  {0pt}{12pt}{6pt}

\renewcommand\normalsize{\fontsize{13}{15.6}\selectfont}
\normalsize

% Márgenes
\usepackage[a4paper, left=3cm, right=3cm, top=2.5cm, bottom=2.5cm]{geometry}

% Interlineado
\setstretch{1.2}

% Encabezado y pie de página
\pagestyle{fancy}
\fancyhf{}
\fancyhead[R]{\includegraphics[width=4cm]{logo.png}} % Logo de la Facultad a la derecha
\fancyfoot[C]{\thepage} % Número de página centrado en el pie de página

\renewcommand{\headrulewidth}{0pt}
\setcounter{tocdepth}{3}
\counterwithout{section}{chapter}
\counterwithin{table}{section}
\numberwithin{figure}{section} % Numeración de figuras por sección
\numberwithin{table}{section} % Numeración de tablas por sección 

% Estilo para las secciones (en negrita)
\titlecontents{section}
  [0em] % Espacio izquierdo
  {\bfseries} % Formato de la etiqueta
  {\thecontentslabel\hspace{1em}} % Formato del número de sección
  {} % Espacio entre la etiqueta y el título
  {\titlerule*[0.5pc]{.}\contentspage} % Formato del título

% Estilo para las subsecciones
\titlecontents{subsection}
  [1.5em] % Espacio izquierdo
  {\normalfont} % Formato de la etiqueta
  {\thecontentslabel\hspace{1em}} % Formato del número de subsección
  {} % Espacio entre la etiqueta y el título
  {\titlerule*[0.5pc]{.}\contentspage} % Formato del título

\titlecontents{subsubsection}
  [3em] % Indentación para subsubsección
  {\thecontentslabel\hspace{1em}} % Formato del número de subsubsección
  {} % Espacio entre la etiqueta y el título
  {} % No se usa prefijo antes del título
  {\titlerule*[0.5pc]{.}\contentspage} % Puntos de conexión y número de página

  
% Definir el formato para las secciones en el TOC (table of contents)
\titlecontents{section}
  [0em] % Espacio izquierdo
  {} % Formato de la etiqueta (vacío para eliminar el número)
  {} % Formato del número de sección (vacío)
  {} % Espacio entre la etiqueta y el título
  {\titlerule*[0.5pc]{.}\contentspage} % Formato del título
  
\renewcommand{\cftsecdotsep}{4} % Ajusta para secciones
\renewcommand{\cftsubsecdotsep}{4} % Ajusta para subsecciones
\renewcommand{\cftsubsubsecdotsep}{4} % Ajusta para subsubsecciones
\newcommand{\fuente}[1]{\fontsize{14}{16}\selectfont\itshape}




%%% Documento %%% 





\begin{document}

% Portada
\begin{titlepage}
    \centering 
    \begin{figure}[h!]
        \centering
        \includegraphics[width=0.5\textwidth]{logo.png}
    \end{figure}
    
    \vspace*{1cm} 
    \Huge
    \textbf{Trabajo Fin de Grado}
    \vspace*{1cm}
    
    \Huge
    
    %Titulo del trabajo
    \textbf{Plantilla para el TFG con \LaTeX}
    
    \vspace{0.5cm}
    \LARGE 
    %Nombre del alumno
    Álvaro Mezquita Martínez
    \vspace{0.5cm} 

    Curso 2024/2025
        
    \vspace{0.5cm}

    Grado en Economía
    
    \vfill

    \raggedleft     
    Tutor:  \makebox[5cm]{\dotfill} %Aquí debes escribir el nombre completo de tu tutor  
    \\
    Departamento:  \makebox[5cm]{\dotfill} %Aquí debes escribir el nombre del departamento pertinente
    
    \vspace{5cm}
\end{titlepage}

% Declaración de originalidad

\newpage
\thispagestyle{empty} % Sin numeración de página en esta página
\noindent {\fontsize{16}{19.2}\selectfont \textbf{Declaración de originalidad y de visto bueno del tutor o tutora}} % "Declaración de originalidad" formateado

\vspace{12pt}

Yo, \makebox[2cm]{\dotfill}, con DNI  \makebox[2cm]{\dotfill}, declaro:  %Eliminar \makebox[]{\dotfill}

\vspace{12pt}

Que este Trabajo Fin de Grado que presento para su evaluación y defensa es original, y que todas las fuentes utilizadas para su realización han sido debidamente citadas en el mismo. 

\vspace{12pt}

Y que cuento con el visto bueno de mi tutor/a para solicitar la defensa en este llamamiento.
\begin{flushright}
    Málaga a  \makebox[2cm]{\dotfill} \\ %Eliminar \makebox[]{\dotfill}
    \vspace{5cm}
     Firmado: \makebox[5cm]{\dotfill} %Eliminar \makebox[]{\dotfill}
\end{flushright}



\newpage
\pagenumbering{roman} % Numeración romana para el índice y resumen

% Índice (NO TOCAR)
\renewcommand{\contentsname}{ÍNDICE} 
\tableofcontents

% Resumen, Palabras Clave y Título en Inglés 
\newpage 
\pagenumbering{arabic} %Cambia la numeracion a romana
\noindent {\fontsize{16}{19.2}\selectfont \textbf{Título en inglés}} % Apartado "Título en ingles" formateado
\addcontentsline{toc}{section}{Título en inglés} % Añadir al índice

\vspace{12pt}

\noindent TFG Template in \LaTeX %Titulo en inglés del trabajo.

\vspace{12pt}

\noindent {\fontsize{16}{19.2}\selectfont \textbf{Resumen}}
\addcontentsline{toc}{section}{Resumen} 

\vspace{12pt}

%Resumen del trabajo
[Estos tres apartados solo deben ocupar una pagina]

\vspace{12pt}

(Aquí debes escribir el resumen de tu TFG. A continuación un texto de ejemplo)

\vspace{12pt}

Lorem ipsum dolor sit amet, consectetur adipiscing elit. Duis bibendum aliquet dolor elementum interdum. In interdum sodales gravida. Sed vulputate nulla feugiat, imperdiet urna non, ultricies libero. Cras in ligula non ipsum mollis tempus. Nullam hendrerit metus ligula, et consectetur libero porta nec. 

\vspace{12pt}

Phasellus fringilla, magna nec tincidunt imperdiet, lectus ipsum congue mi, eu consequat justo arcu nec diam. Morbi interdum sagittis felis nec rhoncus. Pellentesque tristique leo ut purus tristique lobortis. Nunc volutpat imperdiet elit, sit amet pharetra elit tincidunt eget. Cras auctor mattis ipsum a euismod. Nulla rutrum convallis tincidunt. Duis ultricies egestas ipsum lacinia egestas. Vivamus ultricies sapien dignissim augue pellentesque convallis. Nulla id facilisis odio. Nunc a lorem quam. Duis ac leo efficitur, sagittis urna ac, condimentum lectus.

\vspace{12pt}

{\fontsize{16}{19.2}\selectfont \textbf{Palabras clave}}
\addcontentsline{toc}{section}{Palabras Clave} % Añadir al índice

\vspace{12pt}

% Palabras clave de tu trabajo
\LaTeX, TFG, Quantitative methods, Statistics


\newpage  

% A la hora de utilizar \section y \subsection, ten en cuenta que \section debe ir solo y \subsection debe ir junto con el comando \addcontentsline para incluirlo correctamente en el índice

\section{Introducción}

\vspace{12pt}

(Incluir aquí la introducción del trabajo)

\vspace{12pt}

\section{Desarrollo}  

\vspace{12pt}

(Incluir aquí el desarrollo del trabajo)


\subsection*{Subepígrafe} 
\addcontentsline{toc}{subsection}{Subepígrafe} % Añadir al índice


\subsubsection*{Subsubepígrafe} 
\addcontentsline{toc}{subsubsection}{Subsubepígrafe} % Añadir al índice

\begin{center}
    \centering % Centra el caption
    \captionof{figure}{Logo de la facultad} % Nombre de la figura
    \includegraphics[width=\linewidth]{logo.png} % Imagen
    \label{fig:imagen_facultad} % Etiqueta para referenciar
    \par % Termina la alineación centrada
    \raggedright % Restablece la alineación a la izquierda 
    {\fontsize{12}{14}\selectfont\textit{Fuente: Facultad de CC. Económicas y EE. (2017).}} % Fuente en 12 puntos y en itálico
\end{center}

\vspace{12pt}

Este es un ejemplo de referencia de la figura/imagen \ref{fig:imagen_facultad}, y este es un ejemplo para referenciar la bibliografía \citep{tfg_sample}

\vspace{12pt}

Aquí otro ejemplo con un artículo \citep{baccini2007edgeworth}

\vspace{12pt}


\section{Conclusión} 

\vspace{12pt}

(Incluir aquí el conclusión del trabajo)


% Bibliografía (NO TOCAR)

\bibliographystyle{apalike} % Establecer formato APA 7a Edición
\renewcommand{\bibname}{\fontsize{16}{20}\selectfont Bibliografía} % Nombre y formato de la bibliografía
\bibliography{bib_tfg} % Llamada a la bibliografía
\addcontentsline{toc}{section}{Bibliografía} % Añadir al índice
\end{document}
